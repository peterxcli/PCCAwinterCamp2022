\documentclass[a4paper, 12pt]{article}
\usepackage{fontspec}
\usepackage{xeCJK}
\usepackage{mathtools}
\usepackage{xcolor}
\usepackage{listings}
\lstset { %
    language=C++,
    backgroundcolor=\color{black!5}, % set backgroundcolor
    basicstyle=\footnotesize,% basic font setting
}
% \setmainfont{新細明體}
\setCJKmainfont{新細明體}

\XeTeXlinebreaklocale "zh"

\title{輸入標題}
\author{李緒成}

\begin{document}

    \begin{itemize}
        \item 給你一個數列$a_i$,有兩種操作:區間求和;
        $$\sum_{i=l}^{r}(a[i]+=fib[i-l+1])$$\(fib\)是斐波那契數列。
        思路
        \(fib[n] = \frac{\sqrt5}{5}\times [(\frac{1+\sqrt5}{2})^n-(\frac{1-\sqrt5}{2})^n]\)
        
        \item 有關取模、同余、逆元的一些東西:\(p = 1e9 + 9\)
        \item \(383008016^2 \equiv 5 (mod\;p)\)
        \item \(383008016 \equiv \sqrt5 (mod\;p)\)
        \item \(\frac{1}{\sqrt5} \equiv 276601605(mod\;p)\)
        \item \(383008016^{-1} \equiv 276601605(mod\;p)\)
        \item \((1+\sqrt5)/2 \equiv 691504013(mod\;p)\)
        \item \(383008017\times 2^{-1} \equiv 691504013(mod\;p)\)
        \item \((1-\sqrt5)/2 \equiv 308495997(mod\;p)\)
        \item \((p-383008016+1)\times 2^{-1} \equiv 308495997(mod\;p)\)
        
        \item \(fib[n] = 276601605\times [(691504013)^n-(308495997)^n] (mod\;\;p)\)
        \item \(sum = \frac{a}{a-1} \times (a^n - 1) (mod\;\;p) = a^2(a^n-1)(mod\;\;p)=a^{n+2}-a^2(mod\;\;p)\)
        \item 當\(p=1e9+9, a = \{691504013, 308495997\}\)。
        \item 所以本題我們只需要用線段樹lazy標記維護兩個等比數列第一項為一次項的系數即可
    \end{itemize}


\end{document} 